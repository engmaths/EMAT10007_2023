%%%%%%%%%%%%%%%%%%%%%%%%%%%%%%%%%%%%%%%%%%%%%%
% Header
\documentclass[11pt]{report}
\usepackage[english]{babel}
\usepackage[utf8x]{inputenc}
\PassOptionsToPackage{hyphens}{url}\usepackage{hyperref}
\usepackage{graphicx}
\usepackage{fullpage}
\usepackage{nicefrac}
\usepackage[lastexercise]{exercise}
\usepackage[dvipsnames]{xcolor}
\usepackage{amsmath}
\usepackage{enumitem}


 \usepackage{minted}
\makeatletter
\AtBeginEnvironment{minted}{\dontdofcolorbox}
\def\dontdofcolorbox{\renewcommand\fcolorbox[4][]{##4}}
\makeatother


\setlength{\parindent}{0cm}

\renewcommand{\ExerciseHeader}{\large\textbf{\ExerciseName~\ExerciseHeaderNB} - \textbf{\ExerciseTitle}\medskip}

\renewcommand{\ExePartHeader}{\medskip\textbf{\ExePartName\ExePartHeaderNB\ExePartHeaderTitle\medskip}}

\begin{document}


%%%%%%%%%%%%%%%%%%%%%%%%%%%%%%%%%%%%%%%%%%%%%%
\subsubsection*{EMAT10007 -- Introduction to Computer Programming}
\subsection*{\Large Exercises -- Week 5. Data Structures}

%%%%%%%%%%%%%%%%%%%%%%%%%%%%%%%%%%%%%%%%%%%%%%%

\noindent\fbox{%
    \parbox{\textwidth}{%
        \subsection*{Getting Started: Pycharm IDE}
        \subsubsection*{Open PyCharm on linux lab computers}
            \begin{itemize}
                \item{Scroll down to bring up log in screen and log in with your UoB user name and password.}
                \item{Click activities (top left corner) to bring up the side panel.}
                \item{Click the grid of 9 dots to bring up applications.}
                \item{Choose JetBrains PyCharm}
                \item{When prompted about the user agreement click accept and read}            
            \end{itemize}
    }        
}%

\vspace{0.5em}

\noindent\fbox{%
    \parbox{\textwidth}{%
        \subsection*{Create a new project and Python file}
            \begin{itemize}
                \item{Click New project or File $>>$ New project $>>$ Pure python}
                \item{Unselect 'Create a main.py welcome script'}
    	    \item{Note the file location:\\ /home/{\bf UoB\_username}/PycharmProjects/{\bf your\_projectname}/venv \\ where {\bf UoB\_username} is your UoB username and rename {\bf your\_projectname} to be a name of your choice e.g. EMAT10007\_exercises} 
                \item{Right click on the folder icon with project name next to it (top left of window).}
                \item{Choose new $>>$ python file}
                \item{Give your file a name e.g. week\_1\_exercises.py}
            \end{itemize}
    }        
}%
            
\vspace{0.5em}

\noindent\fbox{%
    \parbox{\textwidth}{%
        \subsection*{Write and run code}
        Type some code and click the green play arrow at the top to run.

        \subsubsection*{Save your project}
        File $>>$ Save all to save your wor
    
        \subsubsection*{Open a project you created previously}
        Click File $>>$ Open $>>$ /home/{\bf UoB\_username}/PycharmProjects/{\bf your\_projectname}/venv, Open $>>$ New window 
    }        
}%


\vspace{0.5em}

\noindent\fbox{%
    \parbox{\textwidth}{%
        \subsection*{Rules for naming variables}
	\begin{itemize}
            \item Variable names may contain letters or numbers
            \item Variable names must begin with a letter
            \item Variable names are case sensitive ({\tt time} is not the same as {\tt Time})		
            \item Some {\tt keywords} are reserved by the Python language and cannot be used as variable names. For a full list of keywords reserved by Python, enter the following run the following comand in the editor you are using:
            
    		\vspace{0.5em}
    		{\tt help("keywords")}
    		\vspace{0.5em}
    		
    	\item{Use a consistent naming convention:
            \begin{itemize}
                \item {\tt snake\_case}: lower case letters, words separated by underscore ({\tt \_})
                \item {\tt camel\_Case}: first letter of each word capitalised, excluding first word
                \item {\tt Pascal\_Case}: first letter of each word capitalised
            \end{itemize}
            }
    \end{itemize}
    }        
}%


%%%%%%%%%%%%%%%%%%%%%%%%%%%%%%%%%%%%%%%%%%%%%%%

\begin{Exercise}[title=Lists]
  \Question{Make two lists containing the values {\tt [1,2]} and {\tt [3,4]}.}
  \Question{Change the value 1 to the value 5.}
  \Question{Make a nested list that contains both lists.}
  \Question{Use two loops to print out all the values in the nested list (2x2 matrix) one by one.}
  \Question{Write a program that asks the user to input 5 words (strings) and then creates a list containing each word. Print the list. Print the number
    of characters in the last word.}

  \end{Exercise}


\begin{Exercise}[title=List operations]
  \Question{Create a list called \verb+colours+ that stores the colours
    red, green, and blue. What is the result of \verb+3 * colours+?}
  \Question{Create the list of Booleans such as
    \verb+L = [True, True, False]+.
    What is the result of \verb+all(L)+? What is the result of
    \verb+any(L)+? Now consider the list
    \verb+L = [4, 7, -3, 5, 1, 8]+. Write a program that determines
    whether any element of {\tt L} is negative.}
  \Question{
    Using list comprehension, create lists of the following between 0 and 100:
    \begin{itemize}
    \item odd numbers
    \item multiples of 3
    \item prime numbers (NB: this is quite tricky)
    \end{itemize}
    The answers to the last part is [2, 3, 5, 7, 11, 13, 17, 19, 23, 29, 31, 37, 41, 43, 47, 53, 59, 61, 67, 71, 73, 79, 83, 89, 97].}
  

\end{Exercise}
  
\begin{Exercise}[title=Tuples and dictionaries]
  \Question{Make a tuple named  \verb+fondue_ingredient+
    containing the values ``gruyere'' and ``vacherin''.}
  \Question{Print all the items in the tuple.}
  \Question{Change the value ``gruyere'' to the value ``cheddar''. Does it work? Why?}
  \Question{Consider the colours red, blue, green, black, and white.  Their RBG values are
  (1,0,0), (0,1,0), (0,0,1), (0,0,0), (1,1,1), respectively.  Create a dictionary called
  \verb+colours+ that stores the name of the colours as keys and the RGB values as values.}
  \Question{Using the \verb+colours+ dictionary, run the following code
  	\begin{minted}{python}
  	for v in colours:
  	    print(v)
	\end{minted}
	Are the keys or values printed to the screen?  Now try running the code
	\begin{minted}{python}
  	for v in colours.values():
  	    print(v)
	\end{minted}
	What happens in this case?  
	}
\end{Exercise}


\begin{Exercise}[title=Modelling using data structures]
    
    Table \ref{tab:planets} shows some data about the planets in our solar system. The mass of each planet is shown as a factor which when multiplied by the mass of Earth gives the actual mass of the planet in kg. The mass of Earth can be estimated as $5.9722 \times 10^{24}$ kg. The rotation period is given in units of days (d) or hours (h). 
    \begin{table}
    \begin{center}
    \begin{tabular}{ |l|c|c|c| } 
     \hline
     Planet  & Diameter (km) & Mass  & Rotation period \\
     \hline
     Mercury & 4,878           & 0.06  & 58.65 (d) \\ 
     Venus   & 12,100          & 0.82  & 243 (d) \\  
     Earth   & 12,756          & 1.00  & 23.934 (h) \\ 
     Mars    & 6,794          & 0.11  & 24.623 (h) \\ 
     Jupiter & 142,800          & 317.89& 9.842  (h) \\ 
     Saturn  & 120,000          & 95.17 & 10.233 (h) \\ 
     Uranus  & 52,400          & 14.56 & 16 (h) \\ 
     Neptune & 48,400          & 17.24 & 18 (h) \\ 
     Pluto   & 2,445          & 0.002 & 6.39 (d) \\ 
     \hline
    \end{tabular}
    \vspace{0.5em}
    
    \caption{\label{tab:planets} Planet data taken from: https://www.rmg.co.uk/stories/topics/solar-system-data}
    \end{center}
    \end{table}
    
    % https://www.rmg.co.uk/stories/topics/solar-system-data
    
    % \Question{Write a program that identifies and prints the name and density $\rho$ (in kg m$^{-3}$) of the planet with the lowest density. Density, $\rho = \frac{m}{v}$, where $m$ = mass and $v$ = volume. Assume each planet is a perfect sphere}
    
    \Question{Write a program that identifies and outputs the names and rotation periods of planets with a rotation period shorter than Earth's.}
\end{Exercise}


\begin{Exercise}[title=Sorting lists]

	
  \Question{Write a programme that sorts a list of numbers so that
    the numbers are in ascending order.	Do not use the {\tt sort} method
    or the {\tt sorted} function to do this.}
	
	
\end{Exercise}


\end{document}