%%%%%%%%%%%%%%%%%%%%%%%%%%%%%%%%%%%%%%%%%%%%%%
% Header
\documentclass[11pt]{report}
\usepackage[english]{babel}
\usepackage[utf8x]{inputenc}
%\PassOptionsToPackage{hyphens}{url}
\usepackage{hyperref}
\usepackage{graphicx}
\usepackage{fullpage}
\usepackage{nicefrac}
\usepackage[lastexercise]{exercise}
\usepackage[dvipsnames]{xcolor}
\usepackage{amsmath}


% \usepackage{minted}
\makeatletter
\AtBeginEnvironment{minted}{\dontdofcolorbox}
\def\dontdofcolorbox{\renewcommand\fcolorbox[4][]{##4}}
\makeatother


\setlength{\parindent}{0cm}

\renewcommand{\ExerciseHeader}{\large\textbf{\ExerciseName~\ExerciseHeaderNB} - \textbf{\ExerciseTitle}\medskip}

\renewcommand{\ExePartHeader}{\medskip\textbf{\ExePartName\ExePartHeaderNB\ExePartHeaderTitle\medskip}}

\begin{document}
%%%%%%%%%%%%%%%%%%%%%%%%%%%%%%%%%%%%%%%%%%%%%%
\subsubsection*{EMAT10007 -- Introduction to Computer Programming}
\subsection*{\Large Exercises -- Week 8. Reading, writing, and plotting data}

%%%%%%%%%%%%%%%%%%%%%%%%%%%%%%%%%%%%%%%%%%%%%%%

\noindent\fbox{%
    \parbox{\textwidth}{%
        \subsection*{Getting Started: Pycharm IDE}
        \subsubsection*{Open PyCharm on linux lab computers}
            \begin{itemize}
                \item{Scroll down to bring up log in screen and log in with your UoB user name and password.}
                \item{Click activities (top left corner) to bring up the side panel.}
                \item{Click the grid of 9 dots to bring up applications.}
                \item{Choose JetBrains PyCharm}
                \item{When prompted about the user agreement click accept and read}            
            \end{itemize}
    }        
}%

\vspace{0.5em}

\noindent\fbox{%
    \parbox{\textwidth}{%
        \subsection*{Create a new project and Python file}
            \begin{itemize}
                \item{Click New project or File $>>$ New project $>>$ Pure python}
                \item{Unselect 'Create a main.py welcome script'}
    	    \item{Note the file location:\\ /home/{\bf UoB\_username}/PycharmProjects/{\bf your\_projectname}/venv \\ where {\bf UoB\_username} is your UoB username and rename {\bf your\_projectname} to be a name of your choice e.g. EMAT10007\_exercises} 
                \item{Right click on the folder icon with project name next to it (top left of window).}
                \item{Choose new $>>$ python file}
                \item{Give your file a name e.g. week\_1\_exercises.py}
            \end{itemize}
    }        
}%
            
\vspace{0.5em}

\noindent\fbox{%
    \parbox{\textwidth}{%
        \subsection*{Write and run code}
        Type some code and click the green play arrow at the top to run.

        \subsubsection*{Save your project}
        File $>>$ Save all to save your wor
    
        \subsubsection*{Open a project you created previously}
        Click File $>>$ Open $>>$ /home/{\bf UoB\_username}/PycharmProjects/{\bf your\_projectname}/venv, Open $>>$ New window 
    }        
}%


\vspace{0.5em}

\noindent\fbox{%
    \parbox{\textwidth}{%
        \subsection*{Rules for naming variables}
	\begin{itemize}
            \item Variable names may contain letters or numbers
            \item Variable names must begin with a letter
            \item Variable names are case sensitive ({\tt time} is not the same as {\tt Time})		
            \item Some {\tt keywords} are reserved by the Python language and cannot be used as variable names. For a full list of keywords reserved by Python, enter the following run the following comand in the editor you are using:
            
    		\vspace{0.5em}
    		{\tt help("keywords")}
    		\vspace{0.5em}
    		
    	\item{Use a consistent naming convention:
            \begin{itemize}
                \item {\tt snake\_case}: lower case letters, words separated by underscore ({\tt \_})
                \item {\tt camel\_Case}: first letter of each word capitalised, excluding first word
                \item {\tt Pascal\_Case}: first letter of each word capitalised
            \end{itemize}
            }
    \end{itemize}
    }        
  }%

  \section*{Data}
  
  All data used in these exercises can be found in the {\tt sample\_data} folder which can be downloaded as a zip file from Blackboard. 


\begin{Exercise}[title= Line and scatter graphs] 
\label{Ex:Variables}
	\Question{Create three lists of integers named {\tt x} and {\tt y} with the following values: \\
	{\tt x = [0,2,4,5,8,10,13]}\\
	{\tt y = [1,3,3,3,4,5,6]}\\
        {\tt f = [-3,0,1,0,4,6,7]}\\
	Generate a scatter plot of {\tt y} against  {\tt x} .}\\
	\textbf {Hint:}	Remember to use {\tt plt.show()} to display the graph.
	\Question{On the same axes, create a line plot of
          {\tt f} against {\tt x}.}
       	\Question{Alter your figure so it has the following
	\begin{itemize}
        \item The line for {\tt f} vs {\tt x} is red and has thickness 2
        \item title: {\tt Plot of y,f vs x}
        \item x axis label: {\tt x}
        \item a legend
        \end{itemize}}
	\Question{Save your plot as a .pdf file}

\end{Exercise}




%\newpage

\begin{Exercise}[title= Importing data] \label{Ex:Importing_data}


\Question{Import the data from {\tt hourly\_cycle\_count\_weekend.csv} and plot a line graph of the data with `Time' on the horizontal axis and `Total' on the vertical axis. Label the axes.}

\Question{Import the data in file signal\_data.csv. Plot the data as a scatter graph where the first row is the x (horizontal) data and the second row is the y (vertical) data.}\label{Q:signal}

\Question{Import the data from {\tt temperature\_data.txt} and plot the data with the months on the horizontal axis and the temperature on the vertical axis for each city, as three scatter plots on the same graph. Add a figure legend to show which data set is which and label the axes. \textbf{Hint}: this is not a CSV
file so you will not be able to use the {\tt csv} package to help with importing this data.}\label{Q:temperature}

\end{Exercise}


\begin{Exercise}[title= Importing data for Modelling] \label{Ex:Importing_data_modelling}

Real data is often not in the exact form we want to plot it and some data processing is required before using the data.

\Question{The file `douglas\_data.csv` contains a data set of recorded parameters for a sample of wooden beams. Convert the Bend strength to units of Nm$^{-2}$ and produce a scatter plot of bend strength against knot ratio. Label the axes.}\label{Q:beams}

\end{Exercise}



\begin{Exercise}[title = Exporting data]

  \Question{Consider a rectangular bar of length $L$ and cross-sectional
    area $A$. 
    Hooke's law states that the force needed to stretch a
    material by an amount $\lambda$ is given by $F_{H} = 3E A (\lambda - 1)$.
    However, Hooke's law is a simplification and become inaccurate for
    large values of the stretch $\lambda$.
    The neo-Hookean model says that the force is given by
    $F_{nH} = E A (\lambda - 1 / \lambda^2)$, whereas the Gent
    model for the force is
    \begin{align}
      F_{G} = \frac{EA}{1 - \alpha(\lambda - 1)}\,\left(\lambda - \frac{1}{\lambda^2}\right),
    \end{align}
    where $\alpha \ll 1$ is a constant. Assuming that $A = 10^{-4}$ m$^2$,
    $E = 10^{6}$ Pa, $\alpha = 0.08$, plot $F_H$, $F_{nH}$, and $F_G$ 
    as a function of $\lambda$.  You can assume that $1 \leq \lambda \leq 10$.
    All of the curves should be plotted on the same axes. Make sure the
    axes are labelled and the figure has a legend.}

  \Question{Save the data in a .csv file called \verb+force_stretch.csv+.
    The data should be arranged in columns so that the first column
    are values of the stretch $\lambda$, and the second, third, and fourth
    columns are the forces $F_H$, $F_{nH}$, and $F_{G}$. The first row of the
    .csv file should be text data that describes each column.
    }

\end{Exercise}



\end{document}