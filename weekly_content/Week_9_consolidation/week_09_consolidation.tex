\documentclass[11pt, a4paper]{article}

\usepackage{fullpage}
\usepackage{amsmath, amsfonts}

\begin{document}

\begin{center}
\textbf{EMAT10007 Introduction to Computer Programming \\[0.25em]
Week 09 -- Consolidation Exercise}
\end{center}

\subsubsection*{Problem statement}
In this exercise, you will use Python to simulate how a 
grain of sand sinks in water under gravity.  
Due to the smallness of the grain, its motion is affected by 
random collisions with water molecules.  

\subsubsection*{Requirements and tips}
Your code should be contained in a single Python (.py) file.  
When designing your code, you should think about:
\begin{itemize}
\item{Whether the data types you are using are appropriate and optimal.}
\item{Using if/else/elif statements to control the flow of your code.}
\item{Using loops to carry out repetitive tasks rather than copying-and-pasting code.}
\item{Using functions to break your code into smaller
modules and eliminate code repetition.}
\item{The guidelines on functional programming.}
\item{Giving variables and functions concise and descriptive names.}
\item{Documenting your code so that it is easy to understand how it works.}
\item{Ensuring plots are easy to read and contain all of the necessary information.}
\end{itemize}
A solution to the consolidation exercises will be made available on Friday at 5 pm.  


\subsubsection*{Exercises}

\begin{enumerate}

\item{The radius
and density of the grain can be found in the file
{\tt grain.csv}.  Import this data into Python to
compute the mass of the grain.  The mass can be
computed using the equation
\begin{align}
m = \frac{4}{3} \rho R^3
\end{align}
where $\rho$ is the density in units of kg/m$^3$
and $R$ is the radius in units of m.} 

\item{The motion of the sand grain will be calculated by
first dividing
time into $N$ equally spaced points starting at $t = 0$ and ending at
$t = t_\text{end}$.  If $\Delta t = t_\text{end} / (N-1)$ represents the
spacing between grid points, then we can write any time point
$t_n$ as $t_n = n \Delta t$, where $n = 0, 1, \ldots, N-1$.  
Assuming that
$N = 1000$ and $t_\text{end} = 10^{-3}$ s, 
compute the times $t_n$
and store them in a suitable data structure.}

\item{At each time point $t_n$, the force which acts on the grain
must be calculated.  The total vertical force acting on the
grain at any time is given by
\begin{align}
F = m g + c \label{eqn:Fz}
\end{align}
where $m$ is the mass of the grain (calculated in Question 1), 
$g = 9.8$ m/s 
is the gravitational acceleration,
and $c$ is a collision force from the water molecules.  
The collision force at any time point $t_n$
can be set equal to a random number between 
$-2 \times 10^{-4}$ and $2 \times 10^{-4}$.
Calculate the total vertical force at all of the time points $t_n$;
note that the collision force should be different random number at
each time point.
\textbf{Hint}: You will need to import the {\tt random} package by
adding the code {\tt import random} to the top of your Python file.
A random number between $0$ and $1$ can be generated using the
command {\tt random.random()}.}


\item{Now that the force acting on the grain is known,
the vertical speed $v$ can be calculated by applying Newton's laws of
motion.  
Let $v_n$ denote the downwards speed of the sand grain at time
$t_n$.  Then, the speed 
of the sand grain at time $t_{n+1} = t_n + \Delta t$ is given by
\begin{align}
v_{n+1} = v_{n} + \frac{F \Delta t}{m}, \label{eqn:v}
\end{align}
where $F$ is the force at time $t_n$ that was calculated in
Question 3.  Compute $v_n$ at each time point  $t_n$ up to and including
$t_{N-1} = t_\text{end} = 10^{-3}$~s.  You can assume that the initial speed of the grain
is zero, that is, $v_0 = 0$ m/s.  Plot the velocity as a function of time.
Save this figure as \verb+question4.png+. Ensure that your figure
is appropriately labelled.}

\item{Due to the randomness in the model, each time you run your code,
you will see a different result.  However, the average results should
be more or less the same.  You will now extend your code to
calculate the average speed $V$.  
To determine the average speed $V$,  you must repeatedly calculate the 
speed $v$ and then average the results.

Run 5 simulations by 
generating new forces $F$ at each time point $t_n$ using Equation
\eqref{eqn:Fz} and calculate new speeds using Equation \eqref{eqn:v}.  
Then, for each time $t_n$,
compute the average speed $V_n$ by averaging over the 5 values
of $v_n$ that are obtained from your 5 simulations.  
Create a new figure and 
plot all 5 velocities $v$ and 
the average speed $V$ as a function of time on the same figure.}

\item{If there were no collisions, then the speed of the grain should obey
the equation
\begin{align}
v_e(t) = g t.
\label{eqn:v_exact}
\end{align}
The mean speed $V$ should converge to the
speed $v_e$ predicted by \eqref{eqn:v_exact} as more and more 
simulations are used to compute the mean.  Compute the
mean speed $V$ using $20$, $40$, and $80$ simulations.
Plot all of the average speeds along with the $v_e$ in
a new figure.}

\end{enumerate}

\end{document}