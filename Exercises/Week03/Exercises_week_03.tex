%%%%%%%%%%%%%%%%%%%%%%%%%%%%%%%%%%%%%%%%%%%%%%
% Header
\documentclass[11pt]{report}
\usepackage[english]{babel}
\usepackage[utf8x]{inputenc}
\PassOptionsToPackage{hyphens}{url}\usepackage{hyperref}
\usepackage{graphicx}
\usepackage{fullpage}
\usepackage{nicefrac}
\usepackage[lastexercise]{exercise}
\usepackage[dvipsnames]{xcolor}
\usepackage{amsmath}
\usepackage{enumitem}


\usepackage{minted}
\makeatletter
\AtBeginEnvironment{minted}{\dontdofcolorbox}
\def\dontdofcolorbox{\renewcommand\fcolorbox[4][]{##4}}
\makeatother


\setlength{\parindent}{0cm}

\renewcommand{\ExerciseHeader}{\large\textbf{\ExerciseName~\ExerciseHeaderNB} - \textbf{\ExerciseTitle}\medskip}

\renewcommand{\ExePartHeader}{\medskip\textbf{\ExePartName\ExePartHeaderNB\ExePartHeaderTitle\medskip}}

\begin{document}


%%%%%%%%%%%%%%%%%%%%%%%%%%%%%%%%%%%%%%%%%%%%%%
\subsubsection*{EMAT10007 -- Introduction to Computer Programming}
\subsection*{\Large Exercises -- Week 3. \\ Loops}

\noindent\fbox{%
    \parbox{\textwidth}{%
        \subsection*{Getting Started: How to open software}
        \subsubsection*{PyCharm IDE (Integrated Development Environment)}
        \begin{itemize}
            \item{Open PyCharm on a Linux lab computer}
            \begin{itemize}
                \item{Click on Activities in the top left corner of the screen.}
                \item{Open 'Terminal' from the menu tab.}
            \end{itemize}
            \item{To instead install PyCharm on your personal computer}
            \begin{itemize}
                \item{Click on Activities in the top left corner of the screen.}
                \item{Open 'Terminal' from the menu tab.}
                \item{Type or copy and paste the following:}
    	    \item{Press enter}
            \end{itemize}
        \end{itemize}
        \subsubsection*{Online code editors}
        \begin{itemize}
            \item{Open 'Terminal' from the menu tab.}
            \item{Type or copy and paste the following:}
        \end{itemize}
    }        
}%

\vspace{15}

\noindent\fbox{%
    \parbox{\textwidth}{%
        \subsection*{Getting Started: How to save your work}
        \subsubsection*{PyCharm IDE}
        \begin{itemize}
            \item{Open 'Terminal' from the menu tab.}
        \end{itemize}
        \subsubsection*{Online code editors}
        \begin{itemize}
            \item{Open 'Terminal' from the menu tab.}
        \end{itemize}
    }        
}%

\begin{Exercise}[title=For loops]  \label{Ex:ControlFlow}

{\tt for} loops are used for {\bf definite iteration}, when the number of repetitions is specified explicitly in advance. 

    
    \Question{Use a for loop to cast each value in the sequence [1.5, 1.0, 2.1, 3.8] as an integer and print each integer value.}
    \Question{Find the mistake(s) in the following program, which is meant to sum the first 10 multiples of 5:

        \vspace{0.5 em}

        \begin{verbatim}
        total = 0
        for i in range(1,10)
        total = total + 5 * i
        \end{verbatim}

    Fix the program so that the final value of {\tt total} is 275, and print the value of {\tt total}.
          
    }
    
    \Question{Compute the factorial of 10. Recall that the factorial of an integer $n$ is defined as $n! = n \times (n-1) \times (n-2) \times \ldots 2 \times 1$.}

    \Question{Using a {\tt for} loop and the {\tt break} keyword, determine how many positive cubic numbers are less than 2,000. Recall that cubic number is a number of the form $n^3$ where $n$ is an integer.}

    \Question{Use the {\tt zip} function to sum each pair of elements, taking one element from sequence a = [1.4, 2.2, 2.1, 3.8] and one element from sequence b = [0.1, 1.1, 2.1, 1.2], in the order that they appear in the sequence, and print the result of each addition.}
    \Question{Create a variable and assign it a string value. Using the {\tt zip} function and the {\tt range} function write a loop which prints both each letter and its position in the string.\\ \textbf{Hint:} The Python {\tt len()} function returns the length of a string. \\ {\it {\bf Note:} The Python function, {\tt enumerate} can be used to achieve this operation and avoids the need to define the range of values needed for the counter \\ https://www.w3schools.com/python/ref\_func\_enumerate.asp } \\ Now edit the code to use the {\tt break} keyword to terminate the loop prematurely if the letter is `e` {\it before} printing the letter and its position}
    % \Question{Create a variable and assign it a string value. Using the {\tt zip} function and the {\tt range} function write a loop which prints both each letter and its position in the string. \textbf{Hint:} The {\tt len()} function returns the length of a string. 
    % {\it {\bf Note:} the Python function, {\tt enumerate} can be used to achieve this operation and avoids the need to define the range of values needed for the counter (https://www.w3schools.com/python/ref\_func_enumerate.asp)}    }
\end{Exercise}

\begin{Exercise}[title=While Loops]

{\tt while} loops are used for {\bf indefinite iteration}, the code block repeatedly executes until some condition is met. 

    \Question{Using a while loop determine how many positive cubic numbers are less than 2,000. Recall that cubic number is a number of the form $n^3$ where $n$ is an integer.}
    \Question{Write a program that finds the smallest power of 2 that is greater than 100}
    \Question{The Python function {\tt input()}: 
    \begin{itemize}
        \item displays a string given in the parentheses () to the user
        \item accepts typed input from the user
        \outputs this types input within the program as a string
    \end{itemize} 
    Write a program that prompts the user for a password until a value that matches the password {\tt my\_password123} is given
    }

    \Question{Find the mistake(s) in the following program, which is meant to Find the greatest power of 4 that is smaller than 200:	
        \begin{verbatim}
        exponent = 0
        power = 4 * exponent
        
        while power > 200: 
            result = power
            power = 4 * exponent 
            
        print('largest power = ', result) 
        \end{verbatim}

\end{Exercise}




%\newpage


\begin{Exercise}[title=Choosing an appropriate loop type]

    \Question{Write a program that calculates how many years it would take for the value of a savings account to exceed \pounds 400, if the initial (and only) deposit made is \pounds 100 and the annual interest is 5\%.}
    
    \Question{A ball is dropped (initial velocity $u = 0 ms^{-1}$) and falls towards the ground with acceleration due to gravity of $a = 9.81 ms^{-2}$. It is assumed that no other forces act on the ball so the distance travelled by the ball, $d$ (m), at time $t$ (s), can be found by:
    $$d = ut + \frac{1}{2}at^2$$Print the distance from the start position the ball has fallen at 0.2 s intervals for 2 s, assuming the ball does not reach the ground within this time.}

    \Question{The value of $\pi$ can be approximated using the Leibniz formula:
    \begin{align*}
      \pi_N = \sum_{n = 0}^{N} \frac{8}{(4n+1)(4n+3)}
    \end{align*}
    where $N$ is a large number. Taking the limit as $N \to \infty$
    produces the exact value of $\pi$, but this requires evaluating an infinite
    number of terms, which is impossible on a computer. Therefore, we can
    only approximate the value of $\pi$ by using a finite number of terms
    in the sum. Use this formula to compute approximations to $\pi$
    by taking $N = 100$, $N = 1,000$, and $N = 10,000$. 

    
 
    % Fibonacci numbers are used in the analysis of financial markets (e.g. Fibonacci retracement) and computer algorithms (e.g. Fibonacci search technique, Fibonacci heap data structure) and is found in the patterning of many biological systems (e.g. Nautilus shells, sunflowers and pine cones).
\end{Exercise}


\begin{Exercise}[title=Nested loops]

  \Question{Use two {\tt for} loops to compute the double sum
    \begin{align*}
      S = \sum_{i = 1}^{10} \sum_{j = 0}^{5} j^2(i + j)
    \end{align*}
  }

  \Question{A prime number is a natural number greater than 1 that is not a product of two smaller natural numbers. In other words, a prime number cannot be written as a product of two natural numbers that are both smaller than it. Write a program that prints all prime numbers between 1 and 150. \\ {\bf Hints:}
    \begin{itemize}
        \item Remember the modulo operato, {\tt \%}, gives the remainder when one number is divided by another.
        \item Use two nested loops to cycle through each value, then cycle through the series of possible factors.
    \end{itemize}     
    }
     
  }
   
\end{Exercise}

\begin{Exercise}[title=Text based adventure game]

Use the control structures you have learnt so far to build a text-based adventure game (example below). When you have built your game, get the person sitting next to you to play it. You can also send it to me at hemma.philamore@bristol.ac.uk :)

Example game:

\begin{verbatim}

    print('You enter the castle.')
    print('You see a mysterious figure appear in the print distance.'
    print('Where would you like to go?')

    # Ask user to choose direction
    user_input = input('Choose direction (Options: right/left/backward)')
    
    # Wait until valid response given by user
    while user_input != 'right' and user_input != 'left' and user_input != 'backward' :
        user_input = input('Please enter a valid option: right/left/backward')

    # Act on user input
    if userInput == 'right':
        # Ask user to choose direction
        print('You see a portal)
        user_input = input('Enter the portal? (Options: yes/no)')
        
        # Wait until valid response given by user
        while user_input != 'yes' and user_input != 'no':
            user_input = input('Please enter a valid option: yes/no')
    
        # Act on user input
        if userInput == 'yes':
            print('The portal takes you to a black hole!)
            print('you lose the game!)

        else:
            if ...
      
    elif userInput == 'left':
      print("You find that this door opens into a dead end.")
      
    else:
      if ...
    
\end{verbatim}
    
\end{Exercise}

\end{document}