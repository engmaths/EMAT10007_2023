%%%%%%%%%%%%%%%%%%%%%%%%%%%%%%%%%%%%%%%%%%%%%%
% Header
\documentclass[11pt]{report}
\usepackage[english]{babel}
\usepackage[utf8x]{inputenc}
\PassOptionsToPackage{hyphens}{url}\usepackage{hyperref}
\usepackage{graphicx}
\usepackage{fullpage}
\usepackage{nicefrac}
\usepackage[lastexercise]{exercise}
\usepackage[dvipsnames]{xcolor}
\usepackage{listings}
\usepackage{enumitem}
\graphicspath{ {./img/} }


\begin{document}

\setlength{\parindent}{0cm}

\renewcommand{\ExerciseHeader}{\large\textbf{\ExerciseName~\ExerciseHeaderNB} - \textbf{\ExerciseTitle}\medskip}

\renewcommand{\ExePartHeader}{\medskip\textbf{\ExePartName\ExePartHeaderNB\ExePartHeaderTitle\medskip}}
%%%%%%%%%%%%%%%%%%%%%%%%%%%%%%%%%%%%%%%%%%%%%%
\title{Exercises -- Week 7: Importing Python Files}
\subsubsection*{EMAT10007 -- Introduction to Computer Programming}
\subsection*{\Large Exercises -- Week 7. Reading and Writing Files}


\subsection*{\Large 7.1 Reading and Writing Files}

\subsection*{Essential Questions}


\begin{Exercise}[title= Writing files] \label{Ex:WritingFiles}
	\Question{Write the following table to a new file called samples.txt:
	
	\begin{center}
    \begin{tabular}{ |c|c| } 
     \hline
     Sample & Mass(kg)  \\ 
     \hline
     A  & 0.600  \\ 
     B  & 0.455  \\ 
     C  & 0.550  \\ 
     D  & 0.505  \\ 
     E  & 0.550  \\ 
     \hline
    \end{tabular}
    \end{center}
	
	{\bf Hint:} Remember to close the file!} \label{Q:Table}
	
	\Question{Add the new entries: (Sample F: Mass = 0.505 kg, Sample G: Mass = 0.535 kg) to the table in samples.txt.}

\end{Exercise}

\begin{Exercise}[title= Reading files] \label{Ex:Variables}
	\Question{Write a Python program to read the contents of the table you have just created in samples.txt and print its contents to the console in Spyder.}
	\Question{We can convert the iterable file object returned by {\tt open} as a list of lists by casting using {\tt list(<object\_name>)}.\\ Use this method and then print the third row of the table (excluding the headings).}
	\Question{Print the numerical data in the column Mass as a list of {\tt float} values, shown in grams (g) \\ {\bf Hint:} Exclude the column heading from the list. Convert string data to numerical data.}
	\Question{Print the minimum value of mass in the table ({\bf Hint:} You can use Python's built in {\tt min} function)}
	
	
\end{Exercise}


\begin{Exercise}[title= Reading and writing files] \label{Ex:Reading_Writing_Files}

    \Question{Edit the table in samples.txt so that the column headings are in upper case letters. \\
    {\bf Hint:} You can use Python's built-in {\tt upper} method (example below) which can be used to convert string data to uppercase letters (there is an equivalent {\tt lower} method for converting to lower case).}
    
    \vspace{0.5 em}
    \begin{verbatim}
    txt = "Hello"
    x = txt.upper()
    print(x)
    \end{verbatim}
    
    {\bf $>$$>$ {\tt HELLO}}
    
    \vspace{0.5 em}
    
    
    \Question{ Write a program that:
    \begin{itemize}
        \item opens the file price\_per\_item.csv and prints the contents to display the data
        \item adds another line (you can make up a new entry to the list of foods and prices).
        \item prints the new contents to confirm the new line has been added
    \end{itemize} }\label{Q:food_price}
    
    \Question{Use your answer to Exercise \ref{Ex:Reading_Writing_Files}.\ref{Q:food_price} to write  a function that:
    \begin{itemize}
        \item opens the file price\_per\_item.csv and prints the contents to display the data
        \item asks the user to input a new food item and price  
        \item adds the new line to price\_per\_item.csv
    \end{itemize}
    }
    
    \Question{Store the function in a separate file and call it from your main program.}
    
    

\end{Exercise}

\subsection*{Advanced Questions}

\begin{enumerate}[label=(\Alph*)]


    \item Write a program that edits the Mass of sample B to be 0.485 kg in samples.txt.
    
    \item Write a program that edits the table saved in samples.txt so that all Mass data is rounded to 2 decimal places (the nearest 10g)\\{\bf Hint:} Use the built-in {\tt round} function.
    
    
    
    % \item The Python functions {\tt np.argmax} and {\tt np.argmin} return the {\it element number} of the maximum and minimum values in a list, respectively. 

    % \vspace{0.5 em}
    % \begin{verbatim}
    % nums = np.array([8, 1, 2]) 
    % txt = np.array([`a', `b', `c'])
    % element = nums.argmax() # max value is element 0
    % letter = txt[element]   # txt[0] 
    % print(letter)
    % \end{verbatim}
    % {\bf $>$$>$ {\tt `a'}}
    
    % Write a program that prints the name (letter) of the sample with the lowest mass in the table. 
    
    

        
    

    
\end{enumerate}

\subsection*{\Large 7.2 Imported Modules for Reading and Writing Files:}

\subsection*{Essential Questions}

\begin{Exercise}[title= Writing csv files]\label{Ex:Writing_csv}

\ExeText{}

	\Question{Use the {\tt csv} module to write the table in Exercise \ref{Ex:WritingFiles}.\ref{Q:Table} to a csv file, samples.csv.}
	\Question{Add the new entries: (Sample F: Mass = 0.505 kg, Sample G: Mass = 0.535 kg) to the table in samples.csv.}
	\Question{Us the {\tt csv} module to write a list of numbers to a text file, using spaces as the delimiter.}
	
	
	

\end{Exercise}


\begin{Exercise}[title= Reading csv files]\label{Ex:Reading_csv}

\ExeText{}

	\Question{Read the data in douglas\_data.csv and print it to the Console in Spyder.}
	\Question{Print the maximum value ({\bf Hint:} Python built in {\tt max} function) of  {\tt density} in the data set.}
	\Question{The Python function {\tt sorted()} takes an iterable (e.g. list, string) as an argument and returns it as a sorted list. The argument {\tt reverse} (default value {\tt False} determines whether the items are sorted in ascending or descending (reverse) order. 
    \vspace{0.5 em}
    \begin{verbatim}
    nums = [8, 1, 2]
    print(sorted(nums, reverse=True))
    \end{verbatim}
    {\bf $>$$>$ {\tt [8, 2, 1]}}
    
    Read the data in sample.csv (created in Exercise \ref{Ex:Writing_csv}). Print the values in the Mass column of the imported data in  ascending order. }
    % \Question{Iterate over the files (See lecture slides: {\bf Iterating multiple files}) in folder `a\_folder' and print the file contents. \\{\bf Hint:} Remeber, the function {\tt open} takes the {\it full file path} as the first argument. To join locations as a path you can use {\tt `os.path.join`}
    
    \Question{In statistics, the standard deviation (SD), $\sigma$, is a measure of the amount of variation in a set of values. Low SD indicates that values in a data set tend to be close to the mean, $\mu$, high SD indicates that values are spread over a wider range. The SD is found by taking the square root of the variance, where the variance is the average of the squared difference from the mean $\mu$.
    
    $$\sigma = \sqrt{\frac{\sum_{i=1}^{N} (x_i - \mu)^2}{N}}$$
    
    Write a program that reads the file price\_per\_item.csv and finds the mean and the SD of the values in the 'item\_price' column. 
    
    }

\end{Exercise}

\begin{Exercise}[title= Reading and writing a csv file]\label{Ex:Reading_csv}


\Question{ Write a program that uses the {\tt csv} module to:
    \begin{itemize}
        \item open the file price\_per\_item.csv and prints the contents to display the data
        \item asks the user to input a new food item and price  
        \item adds the new line to price\_per\_item.csv
    \end{itemize} }\label{Q:food_price_csv}


\end{Exercise}


\subsection*{Advanced Questions}




%\begin{Exercise}[title= Reading and writing csv files]\label{Ex:Reading_Writing_csv}



\begin{enumerate}[label=(\Alph*)]
    
    \item  The volume of each sample in samples.csv (created in Exercise \ref{Ex:Writing_csv}) in cm$^3$ is: A = 336, B = 231, C = 350, D = 272, E = 300, F = 312, G = 255. Write a program that:
    {\begin{itemize}
        \item reads the contents of the samples.csv file .
        \item finds the density of each sample in kg/m$^3$.
        \item writes two new columns to the table: volume(cm3) and density(kg/m3)
    \end{itemize}} 
    
    {\bf Hint} Open the file using a mode specifier that lets you read {\it and} write e.g. `'r+`.
    
    {\bf Hint} Numerical data is imported from a  csv file as string data.
    
    {\bf Hint} After reading, the position is at the end of the file. To erase it's contents from some position use {\tt truncate(position)}. To return to the beginning of the file use {\tt seek(0)}. 
    
    \item {\tt sorted()} can be also used to sort one list using the values in another list by using {\tt zip} to group the two lists element wise.
    \vspace{0.5 em}
    \begin{verbatim}
    nums = [8, 1, 2]
    txt = [`a', `b', `c']
    lists = zip(nums, text) 
    s_lists = sorted(lists) # [(1, `b'), (2, `c'), (8, `a')]
    s_nums = [i[0] for i in s_lists] # [1, 2, 3]
    s_txt = [i[1] for i in s_lists] # [`b', `c', `a']
    \end{verbatim}

    Using this example, write a program that:
    \begin{itemize}
        \item asks the user's for a player's name and score.
        \item include the new player and score in the table to that the scores remain in descending order.
        \item writes the sorted table to scores.txt.
        \item prints a message to the Console to tell the user if the new score is the highest in the table e.g. {\tt `[Player name] got a new high score of [score]!'} 
    \end{itemize}
    
    
    
    
    
 
    
\end{enumerate}


\end{document}