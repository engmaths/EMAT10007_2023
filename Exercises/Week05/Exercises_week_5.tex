%%%%%%%%%%%%%%%%%%%%%%%%%%%%%%%%%%%%%%%%%%%%%%
% Header
\documentclass[11pt]{report}
\usepackage[english]{babel}
\usepackage[utf8x]{inputenc}
\usepackage{amsmath}
\usepackage{hyperref}
\usepackage{graphicx}
\usepackage{fullpage}
\usepackage[normalem]{ulem}
\usepackage{listings}
\usepackage[lastexercise]{exercise}
\usepackage{enumitem}
\graphicspath{ {./img/} }

\begin{document}

\setlength{\parindent}{0cm}

\renewcommand{\ExerciseHeader}{\large\textbf{\ExerciseName~\ExerciseHeaderNB} - \textbf{\ExerciseTitle}\medskip}

\renewcommand{\ExePartHeader}{\medskip\textbf{\ExePartName\ExePartHeaderNB\ExePartHeaderTitle\medskip}}

\graphicspath{{../../img/}}

%%%%%%%%%%%%%%%%%%%%%%%%%%%%%%%%%%%%%%%%%%%%%%
\title{Exercises -- Week 5: Importing Python Files}
\subsubsection*{EMAT10007 -- Introduction to Computer Programming}
\subsection*{\Large Exercises -- Week 5. Packages and Modules}



\subsection*{\Large 5.1 Modules}

\begin{Exercise}[title= Importing Python Modules (Essential)]

\begin{verbatim}
                                geometry/
                                |
                                |---- main.py
                                |---- volumes.py
\end{verbatim}



\Question{Create the file system shown above. Leave all .py files empty to begin with.}
\Question{In file volumes.py, write two functions, {\tt sphere} and {\tt cube}. Each function should take one input argument and return the volume of the shape in the name of the function.}
\Question{Edit the content of main.py so that when it is run, it prints the volume of a sphere of radius 3 m}.
\Question{Is there a way to achieve the same operation, using shorter code in main.py?}.
\Question{Add another file, areas.py within the `geometry' folder. In areas.py, write two functions, {\tt sphere} and {\tt cube}. Each function should take one input argument and return the surface area of the shape in the name of the function.}
\Question{Edit the content of main.py so that a variable {\tt A = 5 m} is created and the area and volume of a sphere of radius {\tt A} and a cube of side length {\tt A = 5 m} are printed.}
\end{Exercise}

\begin{Exercise}[title= Importing Python Modules from different sources (Essential)]

\begin{verbatim}
                                building/
                                |
                                |---- main.py
                                |---- funcs.py                        
                                |---- house.py

\end{verbatim}



\Question{Create the file system shown above. Leave all .py files empty to begin with.}
\Question{In file house.py, create three variables {\tt floor} = the floor number of the building as an integer, {\tt width} = the width of the building on this floor in metres as a float, {\tt length} = the length of the building on this floor in metres as a float, and assign them numerical values. }
\Question{In file funcs.py, create a function {\tt ceil} that returns the area of the ceiling using the height and width. }
\Question{Import all contents of house.py, and funcs.py to main.py using {\tt *}. }
\Question{In file main.py, print the following, replacing {\tt <x>} and {\tt <y>} with the area of the value of variable {\tt floor} and the area calculated, respectively:
\begin{verbatim}
The area of the ceiling on floor <x> is <y> m2
\end{verbatim}
}
\Question{In file main.py, calculate the height of the roof using {\tt width} as shown in Figure \ref{fig:roof}, where $\theta = \frac{\pi}{6}$ radians. Import all contents if the Python package {\tt math} to use trigonometric functions e.g. {\tt tan}, and mathematical constants e.g. {\tt pi}.}
\Question{Look at the functions and variables in the {\tt math} package \\ (https://docs.python.org/3/library/math.html) What potential namespace issues could arise in this program? Are any errors generated due to namespace issues when running your program? If so, how can you prevent them from happening? }
\Question{Instead of computing the roof height within the main file, create a function {\tt roof\_height} in funcs.py that returns this value, and call the function in main.py.  Notice how this effects use of functions and variables from {\tt math}. \\ If imported objects (e.g. {\tt math.pi}) are used to define a function, for example {\tt roof\_height}, then the module that the object belongs to (e.g. {\tt math}) must be imported in the file where the function is defined (e.g. funcs.py) rather than where it is called (e.g. main.py).}

\end{Exercise}

\begin{figure}[!h]
        \centering
        \includegraphics[height=5cm]{roof}
        \caption{Roof dimensions}
        \label{fig:roof}
\end{figure}

\subsection*{Advanced Questions}

\begin{enumerate}[label=(\Alph*)]
    
    \item Create your own Python module (.py file). For example, you could store some useful equations that you have learnt/used recently in another unit as Python functions. Or you could simply take some of the functions you wrote in Week 4 and practise assembling them as a Python module and importing them.  
    \item Practise different ways of calling variables, functions and classes from within main.py such as changing the namespace and importing individual functions and/or variables. 
    
\end{enumerate}

\pagebreak

\subsection*{\Large 5.2 Packages}

\begin{verbatim}

                        cubes/
                        |
                        |---- cube.py
                        |---- shapes/
                                |
                                |---- main.py
                                |---- spheres/
                                        |
                                        |---- sphere.py
                                        
\end{verbatim}
\begin{Exercise}[title= Importing from downstream locations (Essential)]
\\
    Create the file system shown above. Leave all .py files empty to begin with.
    \Question{In file sphere.py, create functions {\tt area} and {\tt volume} that take one input argument, the radius.}
    \Question{Edit main.py to print the surface area of a sphere of radius 5 m. {\bf Hint:} There are a number of ways to deal with the need for constants e.g.\ $\pi$. The most reliable way to ensure a constant value is used is to import it from: 
    \begin{itemize}
        \item  an external package e.g. {\tt math} for well known constants like $\pi$
        \item a module created by you to store for program-specific constants - we'll try this out in the next questions ...
    \end{itemize}
    }
    \Question{Create a sub-directory within `spheres' and a file within it, dimensions.py.}
    \Question{Create a variable within dimensions.py called {\tt radius} and give it a numerical value.} 
    \Question{Import the entire contents of dimensions.py to main.py and use {\tt radius} as the input argument to 
    {\tt area} and {\tt volume} to compute and print the area and volume of the sphere.} 
    
\end{Exercise}

\begin{Exercise}[title=Importing from upstream locations (Essential)]\label{Ex:upstream}

    \Question{In file cube.py, create a function {\tt perimeter} that takes one input argument, the side 
    length with a default value of 1 m.  The function {\tt perimeter} should compute the total length of all of the
    edges of the cube}
    \Question{Edit main.py to print the perimeter a cube with side length equal to the value of {\tt radius} in dimensions.py}

\end{Exercise}

\begin{Exercise}[title=Creating a physics package (Essential)]\label{Ex:package}

Create a folder  called {\tt physics} and in this folder create two .py files called
{\tt constants.py} and {\tt equations.py}, as shown below:
\begin{verbatim}

                         current/
                         |---- main.py
                         |---- physics/
                                  |
                                  |---- __init__.py
                                  |---- constants.py
                                  |---- equations.py
\end{verbatim}

The file called {\tt constant.py} will contain a collection of the fundamental
physical constants.  In this file, add the gravitational constant
$G = 6.67 \times 10^{-11}$ m$^3$ kg$^{-1}$ s$^{-2}$,
the speed of light $c = 3.00 \times 10^{8}$ m s$^{-1}$,
Planck's constant $h = 6.63 \times 10^{-34}$ J s$^{-1}$,
the elementary charge $e = 1.60 \times 10^{-19}$ C,
and the mass of an electron $m_e = 9.11 \times 10^{-31}$ kg.
\\[1em]
The file called {\tt equations.py} will contain important equations
from physics.  Define Python functions for 
Newton's law of gravitation
\begin{align}
F = \frac{G m_1 m_2}{r^2},
\end{align}
where $F$ is force (in N), $m_1$ and $m_2$ are two masses (in kg) that are separated by
a distance $r$ (in m); and Einstein's mass-energy equivalence formula
\begin{align}
E = m c^2,
\end{align}
where $E$ is energy (in J), and $m$ is mass (in kg).  
\\[1em]
In your script {\tt main.py}, import your {\tt physics} package and use it to
calculate the energy of an electron.

\end{Exercise}

%\begin{Exercise}[title=Importing from upstream and downstream locations]\label{Ex:upstream_downstream}
%
%    \Question{Create a new .py file upstream of main.py}\label{Q:new_file}
%    \Question{Create a function in the file. }
%    \Question{Call the function from within main.py }
%    \Question{Create a new file downstream of main.py}
%    \Question{Add some variables to the file}
%    \Question{Print the variables within main.py }
%
%\end{Exercise}

\subsection*{Advanced Questions}

\begin{enumerate}[label=(\Alph*)]
    
    \item Python files can be run as a module (imported file) or script (run as program, not imported). When the Python interpreter reads a Python file it sets some variables, then executes the code in the file. One of these variables is {\tt \_\_name\_\_}.  The variable {\tt \_\_name\_\_} takes the value {\tt \_\_main\_\_} if the file is run as a script, and the value is the filename if the file is run as a module. The format shown below is widely used to allow a python file to be run as either a module or a script:
    
    \begin{verbatim}
    if __name__ == "__main__":
       print("equations.py is being used as a script")
    else:
       print("equations.py is being imported and used as a module/package")
    \end{verbatim}
    
    Add this code to the {\tt equations.py} file you created in Exercise 5.  Now, run
    {\tt equations.py} directly (i.e. as a script).  Then run {\tt main.py}.  You
   should notice that different messages are printed to the screen.  The if/else statement
   therefore allows the same file to act differently depending on whether it is being
   used as a script or a module.
    
    % \item \_\_init\_\_.py sometimes contains initialisation code that is run when a package or module is imported. For example, when a Python package is imported, python files within the directory can be automatically imported by adding a line of the form:
    
    % \begin{verbatim}
    % from . import <?>
    % \end{verbatim}
    
    % to \_\_init\_\_.py, where . signifies the current directory and <?> is the name of a python module in the current directory. 
    
    % Edit \_\_init\_\_.py in the spheres directory so that sphere.py is imported when sphere is imported. 
    \item So far we have used {\bf relative imports} to import .py files in other directories to a python programme. The path to the file we want to import is given {\it relative} to the current directory. We can alternatively use {\bf absolute imports} where the path is given relative to the computers home directory. \\
    For example, to import sphere.py to main.py using relative imports, we can edit the Python path {\tt sys.path.append(`../../')}
    
    \begin{verbatim}

                        Desktop/
                        |
                        |---- __init__.py
                        |---- sphere.py
                        |---- my_folder/
                                |
                                |---- my_sub_folder/
                                        |
                                        |---- main.py
                                        
    \end{verbatim}
    
    To edit the Python path using absolute imports instead, we can use: 

    {\bf Windows} 
    \begin{verbatim} sys.path.append(`C:\Desktop\my_folder\my_sub_folder') \end{verbatim}
    
    {\bf Mac, Linux} 
    \begin{verbatim}sys.path.append(`/Users/<YourName>/Desktop/my_folder/my_sub_folder')\end{verbatim}
    
    Note, you must change {\tt <YourName>} to your personal user name. 
    
    Notice, the slashes used to separate the sub-directories lean a different way on Windows than on Mac and Linux.  

    Change the arguments to {\tt sys.path.append} in Exercises \ref{Ex:upstream} and 
    so that the files are imported using the absolute file path. 
    
    \item Create another sub-directory within the `shapes' sub-directory and create your own python package. Create Python file(s) within the directory and practise calling them from within main.py
    
    
\end{enumerate}

\end{document}
