%%%%%%%%%%%%%%%%%%%%%%%%%%%%%%%%%%%%%%%%%%%%%%
% Header
\documentclass[11pt]{report}
\usepackage[english]{babel}
\usepackage[utf8x]{inputenc}
\PassOptionsToPackage{hyphens}{url}\usepackage{hyperref}
\usepackage{graphicx}
\usepackage{fullpage}
\usepackage{nicefrac}
\usepackage[lastexercise]{exercise}
\usepackage[dvipsnames]{xcolor}
\usepackage{amsmath}


\usepackage{minted}
\makeatletter
\AtBeginEnvironment{minted}{\dontdofcolorbox}
\def\dontdofcolorbox{\renewcommand\fcolorbox[4][]{##4}}
\makeatother


\setlength{\parindent}{0cm}

\renewcommand{\ExerciseHeader}{\large\textbf{\ExerciseName~\ExerciseHeaderNB} - \textbf{\ExerciseTitle}\medskip}

\renewcommand{\ExePartHeader}{\medskip\textbf{\ExePartName\ExePartHeaderNB\ExePartHeaderTitle\medskip}}

\begin{document}

%%%%%%%%%%%%%%%%%%%%%%%%%%%%%%%%%%%%%%%%%%%%%
\subsubsection*{Introduction to Computer Programming}
\subsection*{\Large Exercises -- Week 9. NumPy}


\begin{Exercise}[title=Creating arrays and accessing elements (Essential)]

  \Question{Create a variable $a$ that stores the array
    [5, 4, 9, 2, 0, 4, 7, 2].}

  \Question{Print the first entry of $a$ and the last entry of $a$
    (\textbf{Hint}: You can use the index $-1$ to access the last entry
    of an array). The colon operator $:$ can be used to access
    sequential elements of an array. Print the values of
    {\tt a[1:6]} and explain the output.}

  \Question{Change the last entry of $a$ to $-9$ and print the result.
    Now run the command {\tt a[0:3] = 1} and print the result. How has this
    altered the array $a$?}

  \Question{Create an array $r$ that contains 20 random integers between
    $1$ and $9$ (inclusive). Print the result. This array will be used in the
    next question.}

  \Question{(Advanced) \textbf{Logical indexing} provides a quick way to access and modify
    entries in an array that satisfy certain criteria. In this question, we'll use
    logical indexing to replace all of the entries in $r$ that
    are smaller than 5 with 0. First, run the command {\tt idx = r < 5}.
    Print the value of {\tt idx}. Explain the result you see. 
    Now run the command {\tt r[idx] = 0} and print the value of $r$.
    What has happened?}

  \Question{Create a variable $A$ to store the matrix
    \begin{align}
    \begin{pmatrix}
      6 & 2 & 3 \\ 4 & 4 & 1 \\ 8 & 5 & 6
    \end{pmatrix}
    \end{align}
    as a NumPy array.
  }

  \Question{Change the entry in the second row, first column of $A$ to 9. Then
    change the entry in the last row and last column of $A$ to 0. Print the
    updated array $A$. The $n$-th row of $A$ can be accessed using the colon
    operator as {\tt A[n-1,:]}. Similarly, the $m$-th column of $A$ can be
    accessed using {\tt A[:,m-1]}. Use the colon operator to print the entries
    in the second row of $A$.}

  \Question{Create a $2 \times 2$ array of zeros and assign it to a variable $A$. 
    Use the colon
    operator to set the first row of $A$ to 1 and the second row of $A$ to 2.
    \textbf{Hint}: The operation {\tt A[n-1,:]~=~q} will set all of the entries in
    the $n$-th row of $A$ to the value $q$.
    Using a {\tt for} loop, create a $5 \times 5$ matrix where the entries in
    row $i$ are equal to $i$.}
  
\end{Exercise}


\begin{Exercise}[title=Performing operations on NumPy arrays (Essential)]

  \Question{Create two NumPy arrays to store the vectors
    $a = (3, 5, 2)$ and $b = (6, 3, 1)$. Calculate
    $c = a + 2b$. Calculate the dot product $a \cdot b$ using
    the {\tt dot} method or the {\tt dot} function.
    Can you also compute the dot product using
    element-by-element multiplication along with the {\tt np.sum}
    function? Recall that the dot product
    is defined as $a\cdot b = \sum_{i} a_i b_i$.}

  \Question{Create an array called $t$ that contains 500 values between
    0 and 5. Create a
    second array called $y$ that stores the values of $y = t^2 e^{-2 t}$.
    \textbf{Hint}: use the {\tt exp} function to compute the exponential
    of a NumPy array. Find the maximum value of $y$. \textbf{Note}: this is
    a simple way of finding the maximum of a function. 
    (Advanced): Use logical indexing or otherwise to find the value of $t$ at
    which $y$ is maximal.}

  \Question{This question will demonstrate that NumPy can be used to integrate
    functions. Create a NumPy array $x$ that stores 50 values between 0 and 5.
    Create the array $y = x / (x + 1)$. Look up how to use NumPy's {\tt trapz}
    function, which uses the trapezoid rule to approximate integrals.
    Use {\tt trapz} to compute $I = \int_{0}^{5} y(x)\, dx$. 
	The exact value is $I = 5 - \ln(6) = 3.208240530...$
	What happens
    if you repeat the calculation using 500 points between $0$ and $5$?}

  \Question{The table below provides the gravitational acceleration $g$ of each
    of the planets:
    \begin{table}[hp!]
      \centering
      \begin{tabular}{c|c}
        Planet & $g$ [m/s$^2$] \\ \hline
        Mercury & 3.7 \\
        Venus & 8.9 \\
        Earth & 9.8 \\
        Mars & 3.7 \\
        Jupiter & 25\\
        Saturn & 10\\
        Uranus & 8.9\\
        Neptune & 11\\
      \end{tabular}
    \end{table}
    \\
    Use NumPy functions to compute the maximum, minimum, mean, and median values of $g$.}

  \Question{Create NumPy arrays to store the matrices
    \begin{align}
      A = \begin{pmatrix}
        1 & 2 & 3 \\ 3 & 2 & 1 \\ 2 & 4 & 6
      \end{pmatrix},
                                          \quad
                                          B = \begin{pmatrix}
                                            1 & 5 & 0 \\ 0 & 1 & 1 \\ 4 & 3 & 1
                                          \end{pmatrix}
    \end{align}
    Calculate $C = A + 2B$. Then compute $AB$ and $BA$. You should notice that
    $AB \neq BA$.
  }

  \Question{A common operation to perform on matrices is to turn the rows into
    columns and the columns into rows. This is called transposing the matrix.
    Use the function {\tt transpose} to compute the transpose of $A$ and
    print the result.}

  \Question{Solve the linear system of equations $Ax = b$ where
    \begin{align}
      A = \begin{pmatrix}
        1 & 0 & 0 & -1 \\
        1 & -2 & 1 & 0 \\
        0 & 1 & -2 & 1 \\
        2 & 0 & 0 & 1
      \end{pmatrix},
                    \quad
                    b = \begin{pmatrix}
                      0 \\ 1 \\ 1 \\ 2
                    \end{pmatrix}
    \end{align}
    Print the array $x$. Then compute $Ax - b$ and print the result.
  }
  
\end{Exercise}


\begin{Exercise}[title=Weather prediction (Essential)]

  In this example we'll use a \textbf{Markov chain} to create a simple
  model for weather prediction. To start, we will assume that there are
  three states of weather: sunny, cloudy, and rainy. We
  will use the state vector $x = (x_0, x_1, x_2)$ to describe the probabilities
  of the weather being sunny ($x_0$), cloudy ($x_1$), or rainy ($x_2$).
  We use a transition matrix $P$ to describe how the weather changes from one
  day to the next. The entries of the transition matrix, $P_{i,j}$, describe
  the probability of going from state $i$ to state $j$. For this problem,
  we will assume that
  \begin{align}
    P = \begin{pmatrix}
      0.5 & 0.3 & 0.2 \\
      0.4 & 0.2 & 0.4 \\
      0.6 & 0.2 & 0.2
    \end{pmatrix}
  \end{align}
  The entry $P_{1,1} = 0.5$ means there is a 50\% chance that if a day is sunny,
  then the next day will be sunny. Similarly, $P_{3,1} = 0.6$ means there is a
  60\% chance that if a day is rainy, then the next day will be sunny.
  
  \Question{Suppose that today is sunny. Then we can write the state vector
    as $x^{(0)} = (1, 0, 0)$. The weather tomorrow can be predicted by
    computing
    the product $x^{(1)} =  x^{(0)} P$.
    What is the probability that tomorrow will be sunny?}

  \Question{The product $x^{(2)} = x^{(1)} P = x^{(0)} P^2$
    can be used to predict the weather
    in two days. What is the probability that it will rain in two days?}

  \Question{Provide a prediction of the weather for seven days from now. That is,
    compute $x^{(7)}$.}

\end{Exercise}


\end{document}