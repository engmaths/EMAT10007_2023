%%%%%%%%%%%%%%%%%%%%%%%%%%%%%%%%%%%%%%%%%%%%%%
% Header
\documentclass[11pt]{report}
\usepackage[english]{babel}
\usepackage[utf8x]{inputenc}
%\PassOptionsToPackage{hyphens}{url}
\usepackage{hyperref}
\usepackage{graphicx}
\usepackage{fullpage}
\usepackage{nicefrac}
\usepackage[lastexercise]{exercise}
\usepackage[dvipsnames]{xcolor}
\usepackage{amsmath}


\usepackage{minted}
\makeatletter
\AtBeginEnvironment{minted}{\dontdofcolorbox}
\def\dontdofcolorbox{\renewcommand\fcolorbox[4][]{##4}}
\makeatother


\setlength{\parindent}{0cm}

\renewcommand{\ExerciseHeader}{\large\textbf{\ExerciseName~\ExerciseHeaderNB} - \textbf{\ExerciseTitle}\medskip}

\renewcommand{\ExePartHeader}{\medskip\textbf{\ExePartName\ExePartHeaderNB\ExePartHeaderTitle\medskip}}

\begin{document}
%%%%%%%%%%%%%%%%%%%%%%%%%%%%%%%%%%%%%%%%%%%%%%
\subsubsection*{EMAT10007 -- Introduction to Computer Programming}
\subsection*{\Large Exercises -- Week 4. Functions}

 \textbf{Reminder:} The exercise sheets use {\tt <?>} as a placeholder to represent something missing - this could be an operator, a variable name, function name, etc. Multiple {\tt <?>} in the same question are not necessarily representing the same thing.

 \subsection*{\Large Part 1 Functions}
 
\begin{Exercise}[title=Functions (Essential)]
    \Question{Can you complete the {\tt RetSquared} function below by replacing the blank {\tt <?>} with the correct variable name?
    
      \vspace{1em}
      \begin{minted}{python}
        def RetSquared(Number):
            # Returns the square
            Squared = <?> ** 2
            return <?>
    \end{minted}
    \vspace{1em}
    }
    \Question{Use the {\tt RetSquared} function and a {\tt for} loop to print the squares of all integers from 1 to 10.}
    \Question{Can you write a general function for raising numbers to powers? The function should take two arguments {\tt Number} and {\tt Power} and return the number raised to the given power. Remember to give your new function a sensible name.}
    \Question{Use your new general function to print the powers of 2 up to $2^{10}$.}
    \Question{Change this function so that the argument {\tt Power} has the default value {\tt 1}.}
    \Question{Write a new function which has arguments {\tt Number} and {\tt Powers}. If I have inputs {\tt Number=2} and {\tt Powers=[1, 3, 4]}, the function should return the list {\tt [2, 8, 16]}.
    
      How else can you write a function when you don't know the exact number of inputs?}

    \Question{Write a function called {\tt inverse} that computes the
      inverse of a number
      $x$. That is, the function returns $1/x$ if $x \neq 0$.
      If $x = 0$ then your
      function should return the string ``undefined''. Show that
      your function works by computing {\tt inverse(2)} and
      {\tt inverse(0)}.}
    
    \Question{Write a function that swaps the values of two variables.
      Make sure your function returns the new values.
      Include a docstring that explains how to call the function.
      Use the {\tt help} function to print the docstring and read
      about your function. Then,
      create two variables {\tt a = ['red', 'blue', 'green']}
      and {\tt b = \{1, 3, 5\}} and use your function to swap
      their values.}

\end{Exercise}

%%%%%%%%%%%%%%%%%%%%%%%

\begin{Exercise}[title = Modelling with functions (Essential)]

Calculating the area under a curve given by $y(x)$
	is important for many engineering problems.  Calculus tells us
	that the area $A$ under the curve $y(x)$ between the
	points $x = a$ and $x = b$ is given by the integral
	\begin{align}
	A = \int_{a}^{b} y(x)\, dx.
	\end{align}	
	However, in real-world
	engineering problems, the formulas for the curves are often so
	complex that it is impossible to carry out the integration exactly.
	Therefore, the area under the curve must be found approximately.
	One way to approximate the area under the curve is by
	the \href{https://en.wikipedia.org/wiki/Trapezoidal_rule}{trapezium rule}, 
	which divides the area under a curve
	into $N$ trapezoids.  
	According to the trapezium rule, the area under the curve
	can be approximated by the formula
	\begin{align}
	A \approx \frac{1}{2}[y(a) + y(b)] \Delta x + \sum_{i=1}^{N-1} y(x_i) \Delta x,
	\end{align}	
	where $\Delta x = (b - a) / N$ and $x_i = a + i \Delta x$.
	
	\Question{Write a Python program that uses the trapezium rule to calculate
	the area under the curve $y = x^2$ between $x = 0$ and $x = 1$ by
	setting $N = 1$, $N = 10$, and $N = 100$.  Use a Python function to
	define the curve $y(x)$.	
	In this case, the area can be
	calculated exactly as $A = 1/3$.  Use this result to check that your code is
	working correctly.  You should find that as $N$ increases, the value of $A$
	calculated from the trapezium rule becomes more accurate.}
	
	\Question{The density of a hot bar of length $L = 0.1$ m is given by
	the function
	\begin{align}
	y(x) = \alpha e^{-x^2 / \beta^2},
	\end{align}
	where $x$ is a point in the bar,
	 $\alpha = 2$ kg m$^{-1}$ and $\beta = 0.2$~m are constants.  
	Assuming that $0 < x < L$, calculate
	the mass of the bar.  Hint: the mass of the bar can be determined by
	calculating the area under the curve from $x = 0$ to $x = L$. 
	Use the code {\tt from math import *} to access the exponential
	function.  The exponential of a variable {\tt x} can be computed
	using the code {\tt exp(x)}.}
	
	
	
	
	\Question{Write a Python function that calculates the area under any
	curve $y(x)$ between $x = a$ and $x = b$ using the trapezium rule.
	Your Python function should allow the equation for the curve $y(x)$
	to depend on parameters.  
	Hint: functions can be passed to other functions as arguments.}
	
	
\end{Exercise}

%%%%%%%%%%%%%%%%%%%%%%%%%%%%%%%%%%%%%



%%%%%%%%%%%%%%%%%%%%%%%%%%%%%%%%%%%%%%%%%%%%%%%%%%%%%%%%%%
%%%%%%%%%%%%%%%%%%%%%%%%%%%%%%%%%%%%%%%%%%%%%%%%%%%%%%%%%%

\subsection*{\Large Part 2. Functions arguments and scope}

\begin{Exercise}[title=Variadic functions (Essential)]

	Variadic functions can take any number of arguments, just like the print function, which works when used in the following way:\\
	{\tt print("Hello, I am", Age, "years old, born in the month of", Month, ".")}
\Question{Write a function that takes an unknown amount of numbers (integers or floats) and then:
		\begin{itemize}
			\item Calculates the sum of the numbers
			\item Calculates the product of all of the numbers
		\end{itemize}
		and \textbf{returns} these values as opposed to printing them from the function.}

\Question{ Write a function that takes an unknown amount of numbers as its arguments and returns the {\tt min}, {\tt max} and {\tt average} of those numbers.}
\end{Exercise}

%%%%%%%%%%%%%%%%%%%%%%%%%%%%%%%%%%%%%%%%%%%%%%

\begin{Exercise}[title=Variable scope (Essential)]

\textbf{Variable scope.} It is important to understand a little bit about \textbf{scope} -- an important concept in programming. The following exercises will demonstrate how variables are treated depending on whether they are declared inside (local scope) or outside (global scope) of a function.

\Question{Examine the following code and predict the value(s) printed to the
  screen. Then run the code to check your prediction.}
\begin{minted}{python}
  def F():
      print(S)
    
  S = "I hate spam"
  F()
\end{minted}

  \Question{Examine the following code and predict the value(s) printed to the
  screen. Then run the code to check your prediction.}
\begin{minted}{python}
  def F():
      S = "Me too"
      print(S)
  
  S = "I hate spam"
  F()
  print(S)
\end{minted}

  \Question{Examine the following code and predict the value(s) printed to the
  screen. Then run the code to check your prediction.}
\begin{minted}{python}
  def F():
      global S
      S = "Me too"
      print(S)

  S = "I hate spam"
  F()
  print(S)
\end{minted}

% Here we tell the interpreter that the variable {\tt S} is {\tt global} and so a local variable {\tt S} is not being created by Python. Instead, S refers to the \emph{global variable} {\tt S}, or the variable S that was created in the main body of the program, and not in the function {\tt F()}.\\
% 		Check that you understand how the scope of the variable impacts its use.}


 \Question{Examine the following code and predict the value(s) printed to the
  screen. Then run the code to check your prediction.}
% To further illustrate the point, try the following:
  \begin{minted}{python}
    def F():
        T = "I am a variable"
        print(T)
    
    F()
    print(T)
\end{minted}

		% Because {\tt S} is only created and used inside {\tt F()}, and is not passed back to the main body of the program where {\tt F()} is called, then {\tt S} is not defined in the main body of the program.}
		
\end{Exercise}




%%%%%%%%%%%%%%%%%%%%%%%%%%%%%%%%%%%%%%%%%%%%%%%%%%%%%%%%%%%%%%%%%%

\subsection*{\Large Part 3. Recursive functions}

\begin{Exercise}[title= Recursive functions (Essential)]

  \Question{Write a recursive function that computes the
    factorial of a positive integer $n$. Recall that the factorial
    is defined as $n! = n \times (n-1) \times \ldots \times 1$.
    By definition, $0! = 1$. Check your code works by computing
    $5! = 120$. \textbf{Hint:} Notice that $n! = n \times (n-1)!$.}

  \Question{Write a recursive function called  {\tt Fib(n)} that calculates $n$-th number in the Fibonacci sequence using a recursive function.
    The following code
    \begin{minted}{python}
      for i in range(1,15):
          print(Fib(i), end=" ")
    \end{minted}
        should print:
        \\
        {\tt 1~~1~~2~~3~~5~~8~~13~~21~~34~~55~~89~~144~~233~~377}
        \\
        If you are struggling, start by researching how to calculate the Fibonacci sequence. There are lots of tutorials online, so be sure to search for implementations of the Fibonacci function using \emph{recursion}.
		\textbf{Hint:} You will need to identify the ``special cases'' of your program which prevent the function from calling itself infinitely many times.}

\Question{	Read about the relative advantages and disadvantages of using \emph{recursive} functions i.e.\ what happens if {\tt n >= 100}? Is there another version of the Fibonacci function that does not use recursion? Can this be called on large values of {\tt n}?}

  
\end{Exercise}


%%%%%%%%%%%%%%%%%%%%%%%%%%%%%%%%%%%%%%%%%%%%%%%%%%%%%%%%%%%%%%%%%%


\subsection*{\Large Part 4. Advanced questions}

\begin{Exercise}[title = More functions]

  \Question{Write a function that computes the median of three distinct numbers that are input in arbitrary order.  Recall that the median in the number in the middle; it's neither the largest nor the smallest.  For example, the median of 4, 1, and 2 is 2.}

  \Question{Write a function called {\tt sum\_digits} that computes the sum of the digits of an integer. For example {\tt sum\_digits(1234) = 1 + 2 + 3 + 4 = 10}.  There are several ways of doing this; however, if you want a challenge, try to do this using only mathematical operations.}
  
    \Question{In Python it's easy to sort a list of numbers {\tt L} using the {\tt sort} method, that is, by calling {\tt L.sort()}.
  	Write your own function that sorts a list with an arbitrary number of numbers.}
  
  \Question{Write a function which returns the prime factorization of a number.
    \textbf{Note:} Learn about prime factorization here: \url{https://www.mathsisfun.com/prime-factorization.html}}
  
  \end{Exercise}

%\begin{Exercise}[title=Word Scrambler]
%
%    An old internet meme once claimed that if you scrambled the words of a sentence in such a way that the first and last letters remained in place, but the rest of the letters were shuffled, it could still be understood. The following sentence was used to promote the  claim:\\
%
%\emph{``Aoccdrnig to a rscheearch at Cmabrigde Uinervtisy, it deosn't mttaer in waht oredr the ltteers in a wrod are, the olny iprmoetnt tihng is taht the frist and lsat ltteer be at the rghit pclae. The rset can be a toatl mses and you can sitll raed it wouthit porbelm. Tihs is bcuseae the huamn mnid deos not raed ervey lteter by istlef, but the wrod as a wlohe.''}\\
%
%It was later proven incorrect; however, it makes for an interesting programming exercise! \\
%
%How would we write a program which reads in a phrase and prints out the scrambled words? Let's break it down into steps.\\
%
%    \Question{First let's just consider how would we completely scramble a single word? For example, {\tt "scramble"} could become {\tt "lbeacmrs"} or {\tt "toblerone"} could become {\tt "loonberet"}.
%    
%    \textbf{Hint:} Recall the {\tt shuffle} function in the {\tt random} module. However, be aware that {\tt shuffle} rearranges elements in place and so cannot work on strings - you will need to convert to a list.
%    }
%    \Question{So now we can scramble the entire word - how would you adapt this to keep the first and last letters in place? For example, {\tt "scramble"} could become {\tt "smlabcre"} or {\tt "toblerone"} could become {\tt "tnobleroe"}.}
%    \Question{Now rewrite this as a function called {\tt WordScrambler} which takes in an argument {\tt Word}.}
%    \Question{Almost there - can you apply your function to every word in a list of words? For example, the list {\tt ["scramble", "toblerone", "smell", "cheese"]}?}
%    \Question{Finally, rather than have a list of words it'd be nicer to read in a sentence from the user using {\tt input()} and print out the scrambled sentence. How would you do this?
%    }
%    \Question{(Advanced) Enhance your word scrambler. Some ideas include:
%    \begin{itemize}
%        \item Leave punctuation and numbers unchanged.
%        \item Add a flag to the function which switches functionality between scrambling the entire word and leaving the first and last letters in place.
%        \item How would you create a scrambler which only scrambles the position of every other letter?
%    \end{itemize}}
%\end{Exercise}


  
%%%%%%%%%%%%%%%%%%%%%%%%%%%%%%%%%%%%%%%%%%%%%%%%%%%%%%%%%%%%%%%%%%
  

\end{document}